\section{Réalisation}
\subsection{Partie Android}
L'application Android a été développé avec Android Studio et testé sur un Huawei P8 avec la version 5.0.1 d'Android et également sur Sony Xperia Z3 Compact. Le rendu de l'application n'est pas identique au niveau style sur les deux smartphones, mais le bon fonctionnement reste identique.\\
Le projet a permis d'acquérir des connaissances sur les points suivants:
\begin{enumerate}
	\item Découverte d'Android studio
	\item Définition de styles personnalisés
	\item ExpandableListView : Listes contenant des sous-catégories
	\item Communication Bluetooth standard
	\item Utilisation du synthétiseur vocale pour faire parler l'application
	\item Gestion d'un fichier dans la mémoire externe
	\item AlertDialog: Boîte de dialogue\\
\end{enumerate}

\subsection{Partie Arduino}
Concernant la partie Arduino, c'est une plateforme pour l'apprentissage et la réalisation de prototypes de systèmes électroniques. La carte électronique est accompagnée de deux petits circuits imprimés réalisant les fonctionnalités suivantes:
\begin{itemize}
	\item Carte HC-SR04, capteur de distance à ultrasons.
	\item Carte HC-05, carte bluetooth à connecter en UART à un module micro-contrôleur.
\end{itemize}
Pour que le système soit portable, une batterie alimente le tout.