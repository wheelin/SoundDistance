\section{Conclusion}
Le développement de l'application s'est bien passé. Au début du projet, nous avons passé toute une après-midi à définir sur papier les différents écrans de l'application, leur organisation et contenu ainsi que les éléments passés entre les vues. Cette base nous a permis de bien nous répartir le travail à réaliser et à visualiser le résultat désiré.\\\\
Grégory s'est occupé de la programmation du Bluetooth sur l'Arduino ainsi que la gestion du module de mesure par ultrason. Il a également créer un boîtier pour la carte de mesure à l'aide d'une imprimante 3D. Il s'est également occupé, sur Android, de la création de la classe de mesure ainsi que de la vue gérant la prise de mesures. Dans cette partie il a utilisé le synthétiseur vocal du téléphone ainsi qu'une petite animation visuelle pour guider l'utilisateur.\\\\
Emilie a travaillé uniquement sur l'application Android. Elle a défini les différents styles pour les vues, créé la vue avec listant les mesures ainsi que celle permettant de les visualiser. Elle a également implémenté l'écran pour choisir le type de mesure à réaliser et s'est finalement occupée d'implémenter tout ce qui concerne la communication Bluetooth avec le module Arduino.\\\\
Nous sommes satisfaits de notre application, les objectifs que nous nous étions fixés ont été atteints. Nous voulions obtenir une application simple d'utilisation avec un design harmonieux entre les différentes vues. 


